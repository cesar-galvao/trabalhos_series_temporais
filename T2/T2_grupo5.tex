% Options for packages loaded elsewhere
\PassOptionsToPackage{unicode}{hyperref}
\PassOptionsToPackage{hyphens}{url}
\PassOptionsToPackage{dvipsnames,svgnames,x11names}{xcolor}
%
\documentclass[
  letterpaper,
  DIV=11,
  numbers=noendperiod]{scrartcl}

\usepackage{amsmath,amssymb}
\usepackage{iftex}
\ifPDFTeX
  \usepackage[T1]{fontenc}
  \usepackage[utf8]{inputenc}
  \usepackage{textcomp} % provide euro and other symbols
\else % if luatex or xetex
  \usepackage{unicode-math}
  \defaultfontfeatures{Scale=MatchLowercase}
  \defaultfontfeatures[\rmfamily]{Ligatures=TeX,Scale=1}
\fi
\usepackage{lmodern}
\ifPDFTeX\else  
    % xetex/luatex font selection
\fi
% Use upquote if available, for straight quotes in verbatim environments
\IfFileExists{upquote.sty}{\usepackage{upquote}}{}
\IfFileExists{microtype.sty}{% use microtype if available
  \usepackage[]{microtype}
  \UseMicrotypeSet[protrusion]{basicmath} % disable protrusion for tt fonts
}{}
\makeatletter
\@ifundefined{KOMAClassName}{% if non-KOMA class
  \IfFileExists{parskip.sty}{%
    \usepackage{parskip}
  }{% else
    \setlength{\parindent}{0pt}
    \setlength{\parskip}{6pt plus 2pt minus 1pt}}
}{% if KOMA class
  \KOMAoptions{parskip=half}}
\makeatother
\usepackage{xcolor}
\setlength{\emergencystretch}{3em} % prevent overfull lines
\setcounter{secnumdepth}{-\maxdimen} % remove section numbering
% Make \paragraph and \subparagraph free-standing
\ifx\paragraph\undefined\else
  \let\oldparagraph\paragraph
  \renewcommand{\paragraph}[1]{\oldparagraph{#1}\mbox{}}
\fi
\ifx\subparagraph\undefined\else
  \let\oldsubparagraph\subparagraph
  \renewcommand{\subparagraph}[1]{\oldsubparagraph{#1}\mbox{}}
\fi


\providecommand{\tightlist}{%
  \setlength{\itemsep}{0pt}\setlength{\parskip}{0pt}}\usepackage{longtable,booktabs,array}
\usepackage{calc} % for calculating minipage widths
% Correct order of tables after \paragraph or \subparagraph
\usepackage{etoolbox}
\makeatletter
\patchcmd\longtable{\par}{\if@noskipsec\mbox{}\fi\par}{}{}
\makeatother
% Allow footnotes in longtable head/foot
\IfFileExists{footnotehyper.sty}{\usepackage{footnotehyper}}{\usepackage{footnote}}
\makesavenoteenv{longtable}
\usepackage{graphicx}
\makeatletter
\def\maxwidth{\ifdim\Gin@nat@width>\linewidth\linewidth\else\Gin@nat@width\fi}
\def\maxheight{\ifdim\Gin@nat@height>\textheight\textheight\else\Gin@nat@height\fi}
\makeatother
% Scale images if necessary, so that they will not overflow the page
% margins by default, and it is still possible to overwrite the defaults
% using explicit options in \includegraphics[width, height, ...]{}
\setkeys{Gin}{width=\maxwidth,height=\maxheight,keepaspectratio}
% Set default figure placement to htbp
\makeatletter
\def\fps@figure{htbp}
\makeatother

\usepackage{booktabs}
\usepackage{longtable}
\usepackage{array}
\usepackage{multirow}
\usepackage{wrapfig}
\usepackage{float}
\usepackage{colortbl}
\usepackage{pdflscape}
\usepackage{tabu}
\usepackage{threeparttable}
\usepackage{threeparttablex}
\usepackage[normalem]{ulem}
\usepackage{makecell}
\usepackage{xcolor}
\usepackage[auth-lg]{authblk}
\KOMAoption{captions}{tableheading}
\makeatletter
\makeatother
\makeatletter
\makeatother
\makeatletter
\@ifpackageloaded{caption}{}{\usepackage{caption}}
\AtBeginDocument{%
\ifdefined\contentsname
  \renewcommand*\contentsname{Table of contents}
\else
  \newcommand\contentsname{Table of contents}
\fi
\ifdefined\listfigurename
  \renewcommand*\listfigurename{List of Figures}
\else
  \newcommand\listfigurename{List of Figures}
\fi
\ifdefined\listtablename
  \renewcommand*\listtablename{List of Tables}
\else
  \newcommand\listtablename{List of Tables}
\fi
\ifdefined\figurename
  \renewcommand*\figurename{Figure}
\else
  \newcommand\figurename{Figure}
\fi
\ifdefined\tablename
  \renewcommand*\tablename{Table}
\else
  \newcommand\tablename{Table}
\fi
}
\@ifpackageloaded{float}{}{\usepackage{float}}
\floatstyle{ruled}
\@ifundefined{c@chapter}{\newfloat{codelisting}{h}{lop}}{\newfloat{codelisting}{h}{lop}[chapter]}
\floatname{codelisting}{Listing}
\newcommand*\listoflistings{\listof{codelisting}{List of Listings}}
\makeatother
\makeatletter
\@ifpackageloaded{caption}{}{\usepackage{caption}}
\@ifpackageloaded{subcaption}{}{\usepackage{subcaption}}
\makeatother
\makeatletter
\@ifpackageloaded{tcolorbox}{}{\usepackage[skins,breakable]{tcolorbox}}
\makeatother
\makeatletter
\@ifundefined{shadecolor}{\definecolor{shadecolor}{rgb}{.97, .97, .97}}
\makeatother
\makeatletter
\makeatother
\makeatletter
\makeatother
\ifLuaTeX
  \usepackage{selnolig}  % disable illegal ligatures
\fi
\IfFileExists{bookmark.sty}{\usepackage{bookmark}}{\usepackage{hyperref}}
\IfFileExists{xurl.sty}{\usepackage{xurl}}{} % add URL line breaks if available
\urlstyle{same} % disable monospaced font for URLs
\hypersetup{
  pdftitle={Trabalho Prático 2},
  pdfauthor={Ana Carolina Vianna - 18/0097261; César Augusto Galvão - 19/0011572; Yan Flávio Vianna - 14/0166149},
  colorlinks=true,
  linkcolor={blue},
  filecolor={Maroon},
  citecolor={Blue},
  urlcolor={Blue},
  pdfcreator={LaTeX via pandoc}}

\title{Trabalho Prático 2}
\usepackage{etoolbox}
\makeatletter
\providecommand{\subtitle}[1]{% add subtitle to \maketitle
  \apptocmd{\@title}{\par {\large #1 \par}}{}{}
}
\makeatother
\subtitle{Análise de Séries Temporais - 1/2023}
\author{Ana Carolina Vianna - 18/0097261 \and César Augusto Galvão -
19/0011572 \and Yan Flávio Vianna - 14/0166149}
\date{}

\begin{document}
\maketitle
\ifdefined\Shaded\renewenvironment{Shaded}{\begin{tcolorbox}[breakable, sharp corners, enhanced, frame hidden, boxrule=0pt, interior hidden, borderline west={3pt}{0pt}{shadecolor}]}{\end{tcolorbox}}\fi

\renewcommand*\contentsname{Table of contents}
{
\hypersetup{linkcolor=}
\setcounter{tocdepth}{2}
\tableofcontents
}
\newpage{}

\hypertarget{introduuxe7uxe3o-suxe9rie-selecionada-caracteruxedsticas-e-decomposiuxe7uxe3o}{%
\section{Introdução: série selecionada, características e
decomposição}\label{introduuxe7uxe3o-suxe9rie-selecionada-caracteruxedsticas-e-decomposiuxe7uxe3o}}

A série temporal escolhida foi a de número \emph{id} correspondente a
2183. De acordo com a definição do próprio pacote, refere-se a
\emph{Fluid power shipments - hydraulic index}. Foram realizadas medidas
mensais de 1983 a 1992 e o horizonte de previsão requerido é das 18
ocorrências seguintes.

O gráfico da série, com \emph{in} e \emph{out-sample}, é exposto a
seguir.

\includegraphics{T2_grupo5_files/figure-pdf/plot-serie-total-1.pdf}

A série aparenta ter dois períodos, pelo menos: um ciclo anual e outro
que compreende um período maior. No entanto, ao se tentar decompor a
série com múltiplas sazonalidades, obté-se o seguinte:

\begin{itemize}
\tightlist
\item
  \textbf{Adicionando uma componente sazonal com ciclo menor que 1 ano}
  -- uma das componentes sazonais apresenta heteroscedasticidade;
\item
  \textbf{Adicionando uma componente sazonal com ciclo maior que 1 ano}
  -- resíduos apresentam periodicidade ou heteroscedasticidade.
\end{itemize}

Optou-se portanto pela decomposição STL (apesar de os dados terem
inicialmente formado um objeto \texttt{msts}) apenas com a sazonalidade
anual, mas fica evidente que esta decomposição não é adequada quando se
avalia a componente de tendência, que aparenta ainda carregar algum
componente periódico. Os resíduos aparentam um comportamento aleatório e
têm média -0.104, o que é próximo de zero o suficiente considerando a
magnitude dos dados da série. A decomposição é exposta a seguir.

\includegraphics{T2_grupo5_files/figure-pdf/grafico-mstl-1.pdf}

\hypertarget{modelos-arima-seleuxe7uxe3o-transformauxe7uxf5es-e-resuxedduos}{%
\section{Modelos ARIMA: seleção, transformações e
resíduos}\label{modelos-arima-seleuxe7uxe3o-transformauxe7uxf5es-e-resuxedduos}}

\hypertarget{modelo-sem-transformauxe7uxe3o}{%
\subsection{Modelo sem
transformação}\label{modelo-sem-transformauxe7uxe3o}}

\hypertarget{seleuxe7uxe3o}{%
\subsubsection{Seleção}\label{seleuxe7uxe3o}}

Primeiramente, utilizou-se as funções \emph{ndiffs()} e \emph{nsdiffs()}
do pacote \emph{forecast} para identificar quantas diferenças simples e
sazonais seriam necessárias para que a série se tornasse estacionária.
Concluiu-se pelo resultado dessas funções que são necessárias uma
diferenciação simples e uma sazonal. O teste KPSS confirma isso ao não
rejeitar a hipótese nula de estacionariedade da série (com diferenças já
aplicadas) ao nível de 5\% de significância.

\begin{longtable*}{lcc}
\toprule
 & Estatística & p-valor\\
\midrule
\endfirsthead
\multicolumn{3}{@{}l}{\textit{(continued)}}\\
\toprule
 & Estatística & p-valor\\
\midrule
\endhead

\endfoot
\bottomrule
\endlastfoot
\cellcolor{gray!15}{KPSS Test for Level Stationarity} & \cellcolor{gray!15}{0.11} & \cellcolor{gray!15}{0.1}\\*
\end{longtable*}

Prosseguimos com a seleção do melhor modelo ARIMA avaliando os gráficos
de ACF e PACF. O primeiro parece apresentar quebra no primeiro lag
sazonal, enquanto o segundo tem quebra no segundo lag simples.
Entretanto, como não fica nítido um comportamento de queda amortizada,
preferiu-se utilizar outro critério para a seleção do modelo.

\includegraphics{T2_grupo5_files/figure-pdf/acf-pacf-sem-transformacao-1.pdf}

Optou-se pela varredura de combinações de \(p\), \(q\), \(P\) e \(Q\),
com \(d\) e \(D\) fixados em 1, como resultado das diferenciações ja
avaliadas. Utilizando o critério de Akaike corrigido, seleciona-se o
modelo \(\text{ARIMA}(2,1,2)\times(0,1,2)_{12}\) para a série, que
possui o menor escore entre os modelos testados.

Ao se utilizar a função \texttt{auto.arima()}, recebe-se um modelo
sugerido \(\text{ARIMA}(2,1,2)\times(2,1,0)_{12}\), porém com AICc
superior àquele identificado na varredura. Opta-se pelo modelo
selecionado manualmente.

\hypertarget{resuxedduos}{%
\subsubsection{Resíduos}\label{resuxedduos}}

Foram retirados os zeros da inicialização para possibilitar a análise
dos resíduos. Observa-se pelo gráfico que os resíduos são aleatórios e
aparentemente centrados em zero, com variação constante. Além disso,
verifica-se uma distribuição aproximadamente normal, mas com caudas mais
pesadas. Finalmente, o gráfico ACF apresenta que a autocorrelação dos
resíduos está, em sua grande maioria, dentro da banda de confiança, com
exceção de um ponto, que extrapola ligeiramente a margem.

\includegraphics{T2_grupo5_files/figure-pdf/residuos-arima-1.pdf}

Por fim, realiza-se testes de hipótese para independência e normalidade
(o teste KPSS para estacionariedade já foi apresentado) e seus
resultados são apresentados na tabela a seguir. De fato, o teste de
Shapiro-Wilk não rejeita a normalidade da distribuição dos resíduos
apesar de o gráfico QQ apresentar caudas pesadas. Além disso, o teste
Ljung-Box com \emph{lag} igual a 15 também não rejeita a independência
entre os resíduos e, consequentemente, os dados da série.

\begin{longtable*}{lccc}
\toprule
 & Estatística & p-valor & Lag\\
\midrule
\endfirsthead
\multicolumn{4}{@{}l}{\textit{(continued)}}\\
\toprule
 & Estatística & p-valor & Lag\\
\midrule
\endhead

\endfoot
\bottomrule
\endlastfoot
\cellcolor{gray!15}{Box-Ljung test} & \cellcolor{gray!15}{8.90} & \cellcolor{gray!15}{0.88} & \cellcolor{gray!15}{15}\\
Shapiro-Wilk normality test & 0.99 & 0.35 & \\*
\end{longtable*}

\hypertarget{modelo-com-transformauxe7uxe3o}{%
\subsection{Modelo com
transformação}\label{modelo-com-transformauxe7uxe3o}}

\hypertarget{seleuxe7uxe3o-1}{%
\subsubsection{Seleção}\label{seleuxe7uxe3o-1}}

Foi utilizada a função \emph{BoxCox.lambda()} do pacote \emph{forecast}
para decidir de forma automatizada o melhor valor de lambda para a
transformação de Box-Cox. A função sugere um valor de \(\lambda =\)
0.71.

Apesar de haver uma sugestão de transformação, não é possível avaliar
graficamente se houve uma diferença significativa no comportamento da
série temporal excetuando-se a escala, como se pode ver nos eixos dos
gráficos a seguir.

\includegraphics{T2_grupo5_files/figure-pdf/comparacao-transformacao-arima-1.pdf}

Após aplicar a tranformação de Box-Cox na série, utilizou-se as funções
\emph{ndiffs()} e \emph{nsdiffs()} para identificar quantas
diferenciações simples e sazonais seriam necessárias para que a série se
torne estacionária. Concluiu-se que são necessárias uma diferenciações
simples e uma diferenciações sazonal, o que é confirmado pelo resultado
do teste KPSS nos resíduos da série com as diferenças já aplicadas.

\begin{longtable*}{lcc}
\toprule
 & Estatística & p-valor\\
\midrule
\endfirsthead
\multicolumn{3}{@{}l}{\textit{(continued)}}\\
\toprule
 & Estatística & p-valor\\
\midrule
\endhead

\endfoot
\bottomrule
\endlastfoot
\cellcolor{gray!15}{KPSS Test for Level Stationarity} & \cellcolor{gray!15}{0.12} & \cellcolor{gray!15}{0.1}\\*
\end{longtable*}

O gráfico da ACF parece apresentar quebra no primeiro lag sazonal,
enquanto o PACF tem quebra no segundo lag simples. Entretanto, os
gráficos não evidenciam comportamentos claros para a série. Novamente,
os resíduos parecem ter média igual a zero.

\includegraphics{T2_grupo5_files/figure-pdf/acf-pacf-arima-boxcox-1.pdf}

Foram testadas combinações de \(p\), \(q\), \(P\) e \(Q\), com \(d\) e
\(D\) fixados em 1 e, em seguida, selecionou-se o modelo ARIMA que
apresentava menor valor do AICc. Temos, então, que o modelo escolhido
para a série transformada é um
\(\text{ARIMA}(2,1,2)\times(0,1,2)_{12}\), assim como no caso da série
sem transformação. Utilizando-se a função \texttt{auto.arima()}
recebe-se uma sugestão de um modelo \(ARIMA(3,1,1)\times(2,1,0)_{12}\)
mas, assim como ocorre no modelo sem transformação, opta-se pelo modelo
selecionado manualmente por apresentar um AICc menor.

\hypertarget{resuxedduos-1}{%
\subsubsection{Resíduos}\label{resuxedduos-1}}

Foram retirados os zeros da inicialização para seguir com a análise dos
resíduos. O gráfico da série dos resíduos sugere aleatoriedade e o QQ
plot distribuição aproximadamente normal. Por último, o gráfico ACF
mostra que a autocorrelação dos resíduos está dentro da banda de
confiança, com exceção de um ponto que excede um pouco este limite.

\includegraphics{T2_grupo5_files/figure-pdf/residuos-arima-boxcox-1.pdf}

Assim como ocorre para a série não transformada, os testes de
Shapiro-Wilk e Ljung-box com \emph{lag} igual a 15 não apresentam
indicação para rejeição de suas hipóteses nulas. Isto é, pode-se dizer
que a série transformada tem distribuição normal e seus resíduos são
independentes.

\begin{longtable*}{lccc}
\toprule
 & Estatística & p-valor & Lag\\
\midrule
\endfirsthead
\multicolumn{4}{@{}l}{\textit{(continued)}}\\
\toprule
 & Estatística & p-valor & Lag\\
\midrule
\endhead

\endfoot
\bottomrule
\endlastfoot
\cellcolor{gray!15}{Box-Ljung test} & \cellcolor{gray!15}{8.41} & \cellcolor{gray!15}{0.91} & \cellcolor{gray!15}{15}\\
Shapiro-Wilk normality test & 0.98 & 0.23 & \\*
\end{longtable*}

\hypertarget{modelos-ets-seleuxe7uxe3o-transformauxe7uxf5es-e-resuxedduos}{%
\section{Modelos ETS: seleção, transformações e
resíduos}\label{modelos-ets-seleuxe7uxe3o-transformauxe7uxf5es-e-resuxedduos}}

\hypertarget{modelo-sem-transformauxe7uxe3o-1}{%
\subsection{Modelo sem
transformação}\label{modelo-sem-transformauxe7uxe3o-1}}

\hypertarget{seleuxe7uxe3o-2}{%
\subsubsection{Seleção}\label{seleuxe7uxe3o-2}}

Iniciamos a exploração do modelo

\begin{longtable*}{lccc}
\toprule
Modelo & AIC & AICc & BIC\\
\midrule
\endfirsthead
\multicolumn{4}{@{}l}{\textit{(continued)}}\\
\toprule
Modelo & AIC & AICc & BIC\\
\midrule
\endhead

\endfoot
\bottomrule
\endlastfoot
\cellcolor{gray!15}{ETS(M,Ad,M)} & \cellcolor{gray!15}{1761.30} & \cellcolor{gray!15}{1768.36} & \cellcolor{gray!15}{1810.87}\\
ETS(M,M,M) & 1761.94 & 1769.00 & 1811.51\\
\cellcolor{gray!15}{ETS(A,Ad,A)} & \cellcolor{gray!15}{1764.25} & \cellcolor{gray!15}{1771.30} & \cellcolor{gray!15}{1813.81}\\
ETS(M,Ad,A) & 1767.73 & 1774.78 & 1817.29\\
\cellcolor{gray!15}{ETS(M,A,M)} & \cellcolor{gray!15}{1769.04} & \cellcolor{gray!15}{1775.29} & \cellcolor{gray!15}{1815.86}\\
ETS(A,A,A) & 1771.20 & 1777.44 & 1818.01\\*
\end{longtable*}

\includegraphics{T2_grupo5_files/figure-pdf/melhor-fit-ETL-sem-transf-1.pdf}

\hypertarget{resuxedduos-2}{%
\subsubsection{Resíduos}\label{resuxedduos-2}}

\hypertarget{modelo-com-transformauxe7uxe3o-1}{%
\subsection{Modelo com
transformação}\label{modelo-com-transformauxe7uxe3o-1}}

\hypertarget{seleuxe7uxe3o-3}{%
\subsubsection{Seleção}\label{seleuxe7uxe3o-3}}

a série com transformacao

\includegraphics{T2_grupo5_files/figure-pdf/ETS-com-transf-1.pdf}

decomposicao

\includegraphics{T2_grupo5_files/figure-pdf/decomposicao-ets-com-transformacao-1.pdf}

selecao do modelo com transformação

\begin{longtable*}{lccc}
\toprule
Modelo transformado & AIC & AICc & BIC\\
\midrule
\endfirsthead
\multicolumn{4}{@{}l}{\textit{(continued)}}\\
\toprule
Modelo transformado & AIC & AICc & BIC\\
\midrule
\endhead

\endfoot
\bottomrule
\endlastfoot
\cellcolor{gray!15}{ETS(M,Ad,M)} & \cellcolor{gray!15}{1761.30} & \cellcolor{gray!15}{1768.36} & \cellcolor{gray!15}{1810.87}\\
ETS(M,M,M) & 1761.94 & 1769.00 & 1811.51\\
\cellcolor{gray!15}{ETS(Ad,A,A)} & \cellcolor{gray!15}{1764.25} & \cellcolor{gray!15}{1771.30} & \cellcolor{gray!15}{1813.81}\\
ETS(M,Ad,A) & 1767.73 & 1774.78 & 1817.29\\
\cellcolor{gray!15}{ETS(M,A,M)} & \cellcolor{gray!15}{1769.04} & \cellcolor{gray!15}{1775.29} & \cellcolor{gray!15}{1815.86}\\
ETS(A,A,A) & 1771.20 & 1777.44 & 1818.01\\*
\end{longtable*}

OS MODELOS SAO OS MESMO, PODEMO SELECIONAR O SEGUNDO MELHOR

\hypertarget{resuxedduos-3}{%
\subsubsection{Resíduos}\label{resuxedduos-3}}

\hypertarget{estudo-de-desempenho-preditivo}{%
\section{Estudo de desempenho
preditivo}\label{estudo-de-desempenho-preditivo}}

\hypertarget{resultados-da-janela-deslizante}{%
\subsection{Resultados da Janela
Deslizante}\label{resultados-da-janela-deslizante}}

\hypertarget{performance-em-relauxe7uxe3o-aos-horizontes-de-previsuxe3o}{%
\subsection{Performance em relação aos horizontes de
previsão}\label{performance-em-relauxe7uxe3o-aos-horizontes-de-previsuxe3o}}

\hypertarget{arima}{%
\subsubsection{ARIMA}\label{arima}}

\hypertarget{ets}{%
\subsubsection{ETS}\label{ets}}

\hypertarget{resultados}{%
\section{Resultados}\label{resultados}}

apresente em tabelas e gráficos as previsões dos 4 modelos selecionados
e também apresente em uma tabela os resultados de acurácia dos 4 modelos
selecionados e dos modelos benchmarks. Comente os resultados de modo
objetivo;

\hypertarget{apuxeandice}{%
\section{Apêndice}\label{apuxeandice}}

Todo o projeto de composição deste documento pode ser encontrado aqui:
\url{https://github.com/cesar-galvao/trabalhos_series_temporais}



\end{document}
