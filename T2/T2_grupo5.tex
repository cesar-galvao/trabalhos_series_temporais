% Options for packages loaded elsewhere
\PassOptionsToPackage{unicode}{hyperref}
\PassOptionsToPackage{hyphens}{url}
\PassOptionsToPackage{dvipsnames,svgnames,x11names}{xcolor}
%
\documentclass[
  letterpaper,
  DIV=11,
  numbers=noendperiod]{scrartcl}

\usepackage{amsmath,amssymb}
\usepackage{iftex}
\ifPDFTeX
  \usepackage[T1]{fontenc}
  \usepackage[utf8]{inputenc}
  \usepackage{textcomp} % provide euro and other symbols
\else % if luatex or xetex
  \usepackage{unicode-math}
  \defaultfontfeatures{Scale=MatchLowercase}
  \defaultfontfeatures[\rmfamily]{Ligatures=TeX,Scale=1}
\fi
\usepackage{lmodern}
\ifPDFTeX\else  
    % xetex/luatex font selection
\fi
% Use upquote if available, for straight quotes in verbatim environments
\IfFileExists{upquote.sty}{\usepackage{upquote}}{}
\IfFileExists{microtype.sty}{% use microtype if available
  \usepackage[]{microtype}
  \UseMicrotypeSet[protrusion]{basicmath} % disable protrusion for tt fonts
}{}
\makeatletter
\@ifundefined{KOMAClassName}{% if non-KOMA class
  \IfFileExists{parskip.sty}{%
    \usepackage{parskip}
  }{% else
    \setlength{\parindent}{0pt}
    \setlength{\parskip}{6pt plus 2pt minus 1pt}}
}{% if KOMA class
  \KOMAoptions{parskip=half}}
\makeatother
\usepackage{xcolor}
\setlength{\emergencystretch}{3em} % prevent overfull lines
\setcounter{secnumdepth}{-\maxdimen} % remove section numbering
% Make \paragraph and \subparagraph free-standing
\ifx\paragraph\undefined\else
  \let\oldparagraph\paragraph
  \renewcommand{\paragraph}[1]{\oldparagraph{#1}\mbox{}}
\fi
\ifx\subparagraph\undefined\else
  \let\oldsubparagraph\subparagraph
  \renewcommand{\subparagraph}[1]{\oldsubparagraph{#1}\mbox{}}
\fi


\providecommand{\tightlist}{%
  \setlength{\itemsep}{0pt}\setlength{\parskip}{0pt}}\usepackage{longtable,booktabs,array}
\usepackage{calc} % for calculating minipage widths
% Correct order of tables after \paragraph or \subparagraph
\usepackage{etoolbox}
\makeatletter
\patchcmd\longtable{\par}{\if@noskipsec\mbox{}\fi\par}{}{}
\makeatother
% Allow footnotes in longtable head/foot
\IfFileExists{footnotehyper.sty}{\usepackage{footnotehyper}}{\usepackage{footnote}}
\makesavenoteenv{longtable}
\usepackage{graphicx}
\makeatletter
\def\maxwidth{\ifdim\Gin@nat@width>\linewidth\linewidth\else\Gin@nat@width\fi}
\def\maxheight{\ifdim\Gin@nat@height>\textheight\textheight\else\Gin@nat@height\fi}
\makeatother
% Scale images if necessary, so that they will not overflow the page
% margins by default, and it is still possible to overwrite the defaults
% using explicit options in \includegraphics[width, height, ...]{}
\setkeys{Gin}{width=\maxwidth,height=\maxheight,keepaspectratio}
% Set default figure placement to htbp
\makeatletter
\def\fps@figure{htbp}
\makeatother

\usepackage{booktabs}
\usepackage{longtable}
\usepackage{array}
\usepackage{multirow}
\usepackage{wrapfig}
\usepackage{float}
\usepackage{colortbl}
\usepackage{pdflscape}
\usepackage{tabu}
\usepackage{threeparttable}
\usepackage{threeparttablex}
\usepackage[normalem]{ulem}
\usepackage{makecell}
\usepackage{xcolor}
\usepackage[auth-lg]{authblk}
\KOMAoption{captions}{tableheading}
\makeatletter
\makeatother
\makeatletter
\makeatother
\makeatletter
\@ifpackageloaded{caption}{}{\usepackage{caption}}
\AtBeginDocument{%
\ifdefined\contentsname
  \renewcommand*\contentsname{Table of contents}
\else
  \newcommand\contentsname{Table of contents}
\fi
\ifdefined\listfigurename
  \renewcommand*\listfigurename{List of Figures}
\else
  \newcommand\listfigurename{List of Figures}
\fi
\ifdefined\listtablename
  \renewcommand*\listtablename{List of Tables}
\else
  \newcommand\listtablename{List of Tables}
\fi
\ifdefined\figurename
  \renewcommand*\figurename{Figure}
\else
  \newcommand\figurename{Figure}
\fi
\ifdefined\tablename
  \renewcommand*\tablename{Table}
\else
  \newcommand\tablename{Table}
\fi
}
\@ifpackageloaded{float}{}{\usepackage{float}}
\floatstyle{ruled}
\@ifundefined{c@chapter}{\newfloat{codelisting}{h}{lop}}{\newfloat{codelisting}{h}{lop}[chapter]}
\floatname{codelisting}{Listing}
\newcommand*\listoflistings{\listof{codelisting}{List of Listings}}
\makeatother
\makeatletter
\@ifpackageloaded{caption}{}{\usepackage{caption}}
\@ifpackageloaded{subcaption}{}{\usepackage{subcaption}}
\makeatother
\makeatletter
\@ifpackageloaded{tcolorbox}{}{\usepackage[skins,breakable]{tcolorbox}}
\makeatother
\makeatletter
\@ifundefined{shadecolor}{\definecolor{shadecolor}{rgb}{.97, .97, .97}}
\makeatother
\makeatletter
\makeatother
\makeatletter
\makeatother
\ifLuaTeX
  \usepackage{selnolig}  % disable illegal ligatures
\fi
\IfFileExists{bookmark.sty}{\usepackage{bookmark}}{\usepackage{hyperref}}
\IfFileExists{xurl.sty}{\usepackage{xurl}}{} % add URL line breaks if available
\urlstyle{same} % disable monospaced font for URLs
\hypersetup{
  pdftitle={Trabalho Prático 2},
  pdfauthor={Ana Carolina Vianna - 18/0097261; César Augusto Galvão - 19/0011572; Yan Flávio Vianna - 14/0166149},
  colorlinks=true,
  linkcolor={blue},
  filecolor={Maroon},
  citecolor={Blue},
  urlcolor={Blue},
  pdfcreator={LaTeX via pandoc}}

\title{Trabalho Prático 2}
\usepackage{etoolbox}
\makeatletter
\providecommand{\subtitle}[1]{% add subtitle to \maketitle
  \apptocmd{\@title}{\par {\large #1 \par}}{}{}
}
\makeatother
\subtitle{Análise de Séries Temporais - 1/2023}
\author{Ana Carolina Vianna - 18/0097261 \and César Augusto Galvão -
19/0011572 \and Yan Flávio Vianna - 14/0166149}
\date{}

\begin{document}
\maketitle
\ifdefined\Shaded\renewenvironment{Shaded}{\begin{tcolorbox}[sharp corners, interior hidden, borderline west={3pt}{0pt}{shadecolor}, frame hidden, enhanced, boxrule=0pt, breakable]}{\end{tcolorbox}}\fi

\renewcommand*\contentsname{Table of contents}
{
\hypersetup{linkcolor=}
\setcounter{tocdepth}{2}
\tableofcontents
}
\newpage{}

\hypertarget{introduuxe7uxe3o-suxe9rie-selecionada-caracteruxedsticas-e-decomposiuxe7uxe3o}{%
\section{Introdução: série selecionada, características e
decomposição}\label{introduuxe7uxe3o-suxe9rie-selecionada-caracteruxedsticas-e-decomposiuxe7uxe3o}}

A série temporal escolhida foi a de número \emph{id} correspondente a
2183. De acordo com a definição do próprio pacote, refere-se a
\emph{Fluid power shipments - hydraulic index}. Foram realizadas medidas
mensais de 1983 a 1992 e o horizonte de previsão requerido é das 18
ocorrências seguintes.

O gráfico da série, com \emph{in} e \emph{out-sample}, é exposto a
seguir.

\includegraphics{T2_grupo5_files/figure-pdf/plot-serie-total-1.pdf}

A série aparenta ter dois períodos, pelo menos: um ciclo anual e outro
que compreende um período maior. No entanto, ao se tentar decompor a
série com múltiplas sazonalidades, obté-se o seguinte:

\begin{itemize}
\tightlist
\item
  \textbf{Adicionando uma componente sazonal com ciclo menor que 1 ano}
  -- uma das componentes sazonais apresenta heteroscedasticidade;
\item
  \textbf{Adicionando uma componente sazonal com ciclo maior que 1 ano}
  -- resíduos apresentam periodicidade ou heteroscedasticidade.
\end{itemize}

Optou-se portanto pela decomposição STL (apesar de os dados terem
inicialmente formado um objeto \texttt{msts}) apenas com a sazonalidade
anual, mas fica evidente que esta decomposição não é adequada quando se
avalia a componente de tendência, que aparenta ainda carregar algum
componente periódico. Os resíduos aparentam um comportamento aleatório e
têm média -0.104, o que é próximo de zero o suficiente considerando a
magnitude dos dados da série. A decomposição é exposta a seguir.

\includegraphics{T2_grupo5_files/figure-pdf/grafico-mstl-1.pdf}

\hypertarget{modelos-arima-seleuxe7uxe3o-transformauxe7uxf5es-e-resuxedduos}{%
\section{Modelos ARIMA: seleção, transformações e
resíduos}\label{modelos-arima-seleuxe7uxe3o-transformauxe7uxf5es-e-resuxedduos}}

\hypertarget{modelo-sem-transformauxe7uxe3o}{%
\subsection{Modelo sem
transformação}\label{modelo-sem-transformauxe7uxe3o}}

\hypertarget{seleuxe7uxe3o}{%
\subsubsection{Seleção}\label{seleuxe7uxe3o}}

Primeiramente, utilizou-se as funções \emph{ndiffs()} e \emph{nsdiffs()}
do pacote \emph{forecast} para identificar quantas diferenças simples e
sazonais seriam necessárias para que a série se tornasse estacionária.
Concluiu-se pelo resultado dessas funções que são necessárias uma
diferenciação simples e uma sazonal. O teste KPSS confirma isso ao não
rejeitar a hipótese nula de estacionariedade da série (com diferenças já
aplicadas) ao nível de 5\% de significância.

\begin{longtable*}{lcc}
\toprule
 & Estatística & p-valor\\
\midrule
\endfirsthead
\multicolumn{3}{@{}l}{\textit{(continued)}}\\
\toprule
 & Estatística & p-valor\\
\midrule
\endhead

\endfoot
\bottomrule
\endlastfoot
\cellcolor{gray!15}{KPSS Test for Level Stationarity} & \cellcolor{gray!15}{0.11} & \cellcolor{gray!15}{0.1}\\*
\end{longtable*}

Prosseguimos com a seleção do melhor modelo ARIMA avaliando os gráficos
de ACF e PACF. O primeiro parece apresentar quebra no primeiro lag
sazonal, enquanto o segundo tem quebra no segundo lag simples.
Entretanto, como não fica nítido um comportamento de queda amortizada,
preferiu-se utilizar outro critério para a seleção do modelo.

\includegraphics{T2_grupo5_files/figure-pdf/acf-pacf-sem-transformacao-1.pdf}

Optou-se pela varredura de combinações de \(p\), \(q\), \(P\) e \(Q\),
com \(d\) e \(D\) fixados em 1, como resultado das diferenciações ja
avaliadas. Utilizando o critério de Akaike corrigido, seleciona-se o
modelo \(\text{ARIMA}(2,1,2)\times(0,1,2)_{12}\) para a série, que
possui o menor escore entre os modelos testados.

Ao se utilizar a função \texttt{auto.arima()}, recebe-se um modelo
sugerido \(\text{ARIMA}(2,1,2)\times(2,1,0)_{12}\), porém com AICc
superior àquele identificado na varredura. Opta-se pelo modelo
selecionado manualmente.

\hypertarget{resuxedduos}{%
\subsubsection{Resíduos}\label{resuxedduos}}

Foram retirados os zeros da inicialização para possibilitar a análise
dos resíduos. Observa-se pelo gráfico que os resíduos são aleatórios e
aparentemente centrados em zero, com variação constante. Além disso,
verifica-se uma distribuição aproximadamente normal, mas com caudas mais
pesadas. Finalmente, o gráfico ACF apresenta que a autocorrelação dos
resíduos está, em sua grande maioria, dentro da banda de confiança, com
exceção de um ponto, que extrapola ligeiramente a margem.

\includegraphics{T2_grupo5_files/figure-pdf/residuos-arima-1.pdf}

Por fim, realiza-se testes de hipótese para independência e normalidade
(o teste KPSS para estacionariedade já foi apresentado) e seus
resultados são apresentados na tabela a seguir. De fato, o teste de
Shapiro-Wilk não rejeita a normalidade da distribuição dos resíduos
apesar de o gráfico QQ apresentar caudas pesadas. Além disso, o teste
Ljung-Box com \emph{lag} igual a 15 também não rejeita a independência
entre os resíduos e, consequentemente, os dados da série.

\begin{longtable*}{lccc}
\toprule
 & Estatística & p-valor & Lag\\
\midrule
\endfirsthead
\multicolumn{4}{@{}l}{\textit{(continued)}}\\
\toprule
 & Estatística & p-valor & Lag\\
\midrule
\endhead

\endfoot
\bottomrule
\endlastfoot
\cellcolor{gray!15}{Box-Ljung test} & \cellcolor{gray!15}{8.90} & \cellcolor{gray!15}{0.88} & \cellcolor{gray!15}{15}\\
Shapiro-Wilk normality test & 0.99 & 0.35 & \\*
\end{longtable*}

\hypertarget{modelo-com-transformauxe7uxe3o}{%
\subsection{Modelo com
transformação}\label{modelo-com-transformauxe7uxe3o}}

\hypertarget{seleuxe7uxe3o-1}{%
\subsubsection{Seleção}\label{seleuxe7uxe3o-1}}

Foi utilizada a função \emph{BoxCox.lambda()} do pacote \emph{forecast}
para decidir de forma automatizada o melhor valor de lambda para a
transformação de Box-Cox. A função sugere um valor de \(\lambda =\)
0.71.

Apesar de haver uma sugestão de transformação, não é possível avaliar
graficamente se houve uma diferença significativa no comportamento da
série temporal excetuando-se a escala, como se pode ver nos eixos dos
gráficos a seguir.

\includegraphics{T2_grupo5_files/figure-pdf/comparacao-transformacao-arima-1.pdf}

Após aplicar a tranformação de Box-Cox na série, utilizou-se as funções
\emph{ndiffs()} e \emph{nsdiffs()} para identificar quantas
diferenciações simples e sazonais seriam necessárias para que a série se
torne estacionária. Concluiu-se que são necessárias uma diferenciações
simples e uma diferenciações sazonal, o que é confirmado pelo resultado
do teste KPSS nos resíduos da série com as diferenças já aplicadas.

\begin{longtable*}{lcc}
\toprule
 & Estatística & p-valor\\
\midrule
\endfirsthead
\multicolumn{3}{@{}l}{\textit{(continued)}}\\
\toprule
 & Estatística & p-valor\\
\midrule
\endhead

\endfoot
\bottomrule
\endlastfoot
\cellcolor{gray!15}{KPSS Test for Level Stationarity} & \cellcolor{gray!15}{0.12} & \cellcolor{gray!15}{0.1}\\*
\end{longtable*}

O gráfico da ACF parece apresentar quebra no primeiro lag sazonal,
enquanto o PACF tem quebra no segundo lag simples. Entretanto, os
gráficos não evidenciam comportamentos claros para a série. Novamente,
os resíduos parecem ter média igual a zero.

\includegraphics{T2_grupo5_files/figure-pdf/acf-pacf-arima-boxcox-1.pdf}

Foram testadas combinações de \(p\), \(q\), \(P\) e \(Q\), com \(d\) e
\(D\) fixados em 1 e, em seguida, selecionou-se o modelo ARIMA que
apresentava menor valor do AICc. Temos, então, que o modelo escolhido
para a série transformada é um
\(\text{ARIMA}(2,1,2)\times(0,1,2)_{12}\), assim como no caso da série
sem transformação. Utilizando-se a função \texttt{auto.arima()}
recebe-se uma sugestão de um modelo \(ARIMA(3,1,1)\times(2,1,0)_{12}\)
mas, assim como ocorre no modelo sem transformação, opta-se pelo modelo
selecionado manualmente por apresentar um AICc menor.

\hypertarget{resuxedduos-1}{%
\subsubsection{Resíduos}\label{resuxedduos-1}}

Foram retirados os zeros da inicialização para seguir com a análise dos
resíduos. O gráfico da série dos resíduos sugere aleatoriedade e o QQ
plot distribuição aproximadamente normal. Por último, o gráfico ACF
mostra que a autocorrelação dos resíduos está dentro da banda de
confiança, com exceção de um ponto que excede um pouco este limite.

\includegraphics{T2_grupo5_files/figure-pdf/residuos-arima-boxcox-1.pdf}

Assim como ocorre para a série não transformada, os testes de
Shapiro-Wilk e Ljung-box com \emph{lag} igual a 15 não apresentam
indicação para rejeição de suas hipóteses nulas. Isto é, pode-se dizer
que a série transformada tem distribuição normal e seus resíduos são
independentes.

\begin{longtable*}{lccc}
\toprule
 & Estatística & p-valor & Lag\\
\midrule
\endfirsthead
\multicolumn{4}{@{}l}{\textit{(continued)}}\\
\toprule
 & Estatística & p-valor & Lag\\
\midrule
\endhead

\endfoot
\bottomrule
\endlastfoot
\cellcolor{gray!15}{Box-Ljung test} & \cellcolor{gray!15}{8.41} & \cellcolor{gray!15}{0.91} & \cellcolor{gray!15}{15}\\
Shapiro-Wilk normality test & 0.98 & 0.23 & \\*
\end{longtable*}

\hypertarget{modelos-ets-seleuxe7uxe3o-transformauxe7uxf5es-e-resuxedduos}{%
\section{Modelos ETS: seleção, transformações e
resíduos}\label{modelos-ets-seleuxe7uxe3o-transformauxe7uxf5es-e-resuxedduos}}

JUSTIFICAR USO DO AAA EM VEZ DO TOP AICC

\hypertarget{modelo-sem-transformauxe7uxe3o-1}{%
\subsection{Modelo sem
transformação}\label{modelo-sem-transformauxe7uxe3o-1}}

\hypertarget{seleuxe7uxe3o-2}{%
\subsubsection{Seleção}\label{seleuxe7uxe3o-2}}

Iniciamos a exploração do modelo

\begin{longtable*}{lccc}
\toprule
Modelo & AIC & AICc & BIC\\
\midrule
\endfirsthead
\multicolumn{4}{@{}l}{\textit{(continued)}}\\
\toprule
Modelo & AIC & AICc & BIC\\
\midrule
\endhead

\endfoot
\bottomrule
\endlastfoot
\cellcolor{gray!15}{ETS(M,Ad,M)} & \cellcolor{gray!15}{1761.30} & \cellcolor{gray!15}{1768.36} & \cellcolor{gray!15}{1810.87}\\
ETS(M,M,M) & 1761.94 & 1769.00 & 1811.51\\
\cellcolor{gray!15}{ETS(A,Ad,A)} & \cellcolor{gray!15}{1764.25} & \cellcolor{gray!15}{1771.30} & \cellcolor{gray!15}{1813.81}\\
ETS(M,Ad,A) & 1767.73 & 1774.78 & 1817.29\\
\cellcolor{gray!15}{ETS(M,A,M)} & \cellcolor{gray!15}{1769.04} & \cellcolor{gray!15}{1775.29} & \cellcolor{gray!15}{1815.86}\\
ETS(A,A,A) & 1771.20 & 1777.44 & 1818.01\\*
\end{longtable*}

\includegraphics{T2_grupo5_files/figure-pdf/melhor-fit-ETL-sem-transf-1.pdf}

\hypertarget{resuxedduos-2}{%
\subsubsection{Resíduos}\label{resuxedduos-2}}

\hypertarget{modelo-com-transformauxe7uxe3o-1}{%
\subsection{Modelo com
transformação}\label{modelo-com-transformauxe7uxe3o-1}}

\hypertarget{seleuxe7uxe3o-3}{%
\subsubsection{Seleção}\label{seleuxe7uxe3o-3}}

a série com transformacao

\includegraphics{T2_grupo5_files/figure-pdf/ETS-com-transf-1.pdf}

decomposicao

\includegraphics{T2_grupo5_files/figure-pdf/decomposicao-ets-com-transformacao-1.pdf}

selecao do modelo com transformação

\begin{longtable*}{lccc}
\toprule
Modelo transformado & AIC & AICc & BIC\\
\midrule
\endfirsthead
\multicolumn{4}{@{}l}{\textit{(continued)}}\\
\toprule
Modelo transformado & AIC & AICc & BIC\\
\midrule
\endhead

\endfoot
\bottomrule
\endlastfoot
\cellcolor{gray!15}{ETS(M,Ad,M)} & \cellcolor{gray!15}{1761.30} & \cellcolor{gray!15}{1768.36} & \cellcolor{gray!15}{1810.87}\\
ETS(M,M,M) & 1761.94 & 1769.00 & 1811.51\\
\cellcolor{gray!15}{ETS(Ad,A,A)} & \cellcolor{gray!15}{1764.25} & \cellcolor{gray!15}{1771.30} & \cellcolor{gray!15}{1813.81}\\
ETS(M,Ad,A) & 1767.73 & 1774.78 & 1817.29\\
\cellcolor{gray!15}{ETS(M,A,M)} & \cellcolor{gray!15}{1769.04} & \cellcolor{gray!15}{1775.29} & \cellcolor{gray!15}{1815.86}\\
ETS(A,A,A) & 1771.20 & 1777.44 & 1818.01\\*
\end{longtable*}

MESMO MODELO, MAS COM TRANSFORMAÇÃO

\hypertarget{resuxedduos-3}{%
\subsubsection{Resíduos}\label{resuxedduos-3}}

\hypertarget{estudo-de-desempenho-preditivo}{%
\section{Estudo de desempenho
preditivo}\label{estudo-de-desempenho-preditivo}}

Para realizar a análise do desempenho preditivo usando uma abordagem de
janela deslizante, o estudo considera uma janela de tamanho \(n-14\) e
calcula os erros de previsão para horizontes de até 5 períodos.
Utilizando os modelos previamente mencionados para criar as funções de
previsão, Os resultados são apresentados em um gráfico e uma tabela,
mostrando os erros absolutos para cada horizonte de previsão.

\hypertarget{resultados-da-janela-deslizante}{%
\subsection{Resultados da Janela
Deslizante}\label{resultados-da-janela-deslizante}}

\begin{longtable*}{ccccc}
\toprule
 & ARIMA & ETS & ARIMA Transformada & ETS Transformada\\
\midrule
\endfirsthead
\multicolumn{5}{@{}l}{\textit{(continued)}}\\
\toprule
 & ARIMA & ETS & ARIMA Transformada & ETS Transformada\\
\midrule
\endhead

\endfoot
\bottomrule
\endlastfoot
\cellcolor{gray!15}{h=1} & \cellcolor{gray!15}{130.701} & \cellcolor{gray!15}{126.428} & \cellcolor{gray!15}{122.244} & \cellcolor{gray!15}{124.525}\\
h=2 & 133.301 & 128.668 & 144.609 & 136.707\\
\cellcolor{gray!15}{h=3} & \cellcolor{gray!15}{128.709} & \cellcolor{gray!15}{126.408} & \cellcolor{gray!15}{163.981} & \cellcolor{gray!15}{167.741}\\
h=4 & 136.742 & 128.690 & 180.380 & 181.928\\
\cellcolor{gray!15}{h=5} & \cellcolor{gray!15}{173.517} & \cellcolor{gray!15}{161.675} & \cellcolor{gray!15}{210.707} & \cellcolor{gray!15}{210.361}\\*
\end{longtable*}

VER PQ A ETS\_TRANSF ESTÁ DANDO TUDO NA

\hypertarget{performance-em-relauxe7uxe3o-aos-horizontes-de-previsuxe3o}{%
\subsection{Performance em relação aos horizontes de
previsão}\label{performance-em-relauxe7uxe3o-aos-horizontes-de-previsuxe3o}}

\includegraphics{T2_grupo5_files/figure-pdf/unnamed-chunk-1-1.pdf}

ARRUMAR A LEGENDA E OS VALORES E DEPOIS FAZER A ANÁLISE

\hypertarget{gruxe1ficos-da-previsuxe3o-pontual-e-da-previsuxe3o-intervalar-dos-4-modelos-selecionados}{%
\section{Gráficos da previsão pontual e da previsão intervalar dos 4
modelos
selecionados}\label{gruxe1ficos-da-previsuxe3o-pontual-e-da-previsuxe3o-intervalar-dos-4-modelos-selecionados}}

\includegraphics{T2_grupo5_files/figure-pdf/previsao-pontual-1.pdf}

\begin{verbatim}
         Point Forecast    Lo 80    Hi 80    Lo 95    Hi 95
Sep 1992       4795.360 4568.189 5022.531 4447.932 5142.788
Oct 1992       5016.382 4763.010 5269.753 4628.883 5403.880
Nov 1992       4541.278 4248.643 4833.913 4093.731 4988.825
Dec 1992       4391.441 4032.320 4750.562 3842.213 4940.669
Jan 1993       4848.159 4440.260 5256.059 4224.331 5471.988
Feb 1993       4905.299 4467.187 5343.411 4235.264 5575.333
Mar 1993       5329.729 4861.354 5798.103 4613.411 6046.046
Apr 1993       5203.385 4700.475 5706.295 4434.251 5972.520
May 1993       5052.665 4518.132 5587.198 4235.167 5870.163
Jun 1993       5245.419 4683.930 5806.909 4386.695 6104.144
Jul 1993       4859.079 4271.730 5446.428 3960.806 5757.352
Aug 1993       5014.812 4401.327 5628.297 4076.568 5953.057
Sep 1993       5125.554 4459.124 5791.984 4106.338 6144.771
Oct 1993       5328.565 4627.990 6029.140 4257.129 6400.001
Nov 1993       4908.795 4172.038 5645.551 3782.023 6035.566
Dec 1993       4788.711 4010.224 5567.197 3598.119 5979.303
Jan 1994       5210.324 4393.633 6027.015 3961.303 6459.345
Feb 1994       5256.719 4407.774 6105.663 3958.370 6555.067
\end{verbatim}

\includegraphics{T2_grupo5_files/figure-pdf/previsao-pontual-2.pdf}

\begin{verbatim}
         Point Forecast    Lo 80    Hi 80    Lo 95    Hi 95
Sep 1992       4799.752 4565.169 5037.686 4442.368 5164.973
Oct 1992       5035.236 4767.608 5307.025 4627.652 5452.555
Nov 1992       4537.145 4237.281 4842.830 4080.957 5006.952
Dec 1992       4379.433 4016.236 4751.524 3827.684 4951.992
Jan 1993       4854.191 4428.229 5291.205 4207.361 5526.884
Feb 1993       4922.088 4462.585 5394.299 4224.665 5649.249
Mar 1993       5364.435 4861.280 5881.571 4600.788 6160.802
Apr 1993       5235.766 4699.996 5787.823 4423.227 6086.423
May 1993       5077.223 4512.969 5660.159 4222.156 5976.018
Jun 1993       5275.453 4676.146 5895.059 4367.467 6230.952
Jul 1993       4867.736 4256.482 5501.964 3942.656 5846.601
Aug 1993       5035.622 4391.144 5704.808 4060.475 6068.613
Sep 1993       5155.398 4452.819 5886.738 4093.156 6284.981
Oct 1993       5372.068 4624.761 6150.631 4242.499 6574.823
Nov 1993       4923.688 4159.707 5723.503 3770.673 6160.632
Dec 1993       4797.716 3998.702 5637.111 3593.158 6096.874
Jan 1994       5246.596 4386.318 6149.642 3949.353 6644.026
Feb 1994       5302.608 4406.473 6244.738 3951.961 6761.014
\end{verbatim}

\includegraphics{T2_grupo5_files/figure-pdf/previsao-pontual-3.pdf}

\begin{verbatim}
         Point Forecast    Lo 80    Hi 80    Lo 95    Hi 95
Sep 1992       4613.023 4391.648 4834.399 4274.459 4951.588
Oct 1992       4808.078 4557.385 5058.771 4424.676 5191.480
Nov 1992       4437.827 4153.583 4722.072 4003.113 4872.541
Dec 1992       4337.828 4016.796 4658.859 3846.853 4828.803
Jan 1993       4746.470 4386.219 5106.721 4195.513 5297.426
Feb 1993       4771.433 4370.152 5172.715 4157.726 5385.140
Mar 1993       5198.160 4754.516 5641.804 4519.665 5876.655
Apr 1993       5139.633 4652.663 5626.602 4394.877 5884.388
May 1993       4950.914 4419.944 5481.884 4138.865 5762.962
Jun 1993       5019.380 4443.961 5594.799 4139.353 5899.408
Jul 1993       4587.096 3966.960 5207.232 3638.680 5535.512
Aug 1993       4840.204 4175.228 5505.179 3823.211 5857.196
Sep 1993       4844.615 4134.766 5554.463 3758.995 5930.234
Oct 1993       5024.388 4269.787 5778.989 3870.325 6178.451
Nov 1993       4639.864 3840.678 5439.050 3417.615 5862.114
Dec 1993       4526.533 3682.995 5370.072 3236.453 5816.614
Jan 1994       4922.724 4035.118 5810.330 3565.247 6280.201
Feb 1994       4936.057 4004.711 5867.404 3511.686 6360.429
\end{verbatim}

\includegraphics{T2_grupo5_files/figure-pdf/previsao-pontual-4.pdf}

\begin{verbatim}
         Point Forecast    Lo 80    Hi 80    Lo 95    Hi 95
Sep 1992       4631.450 4406.500 4859.593 4288.733 4981.634
Oct 1992       4819.800 4560.493 5083.189 4424.909 5224.240
Nov 1992       4426.133 4138.592 4719.160 3988.650 4876.449
Dec 1992       4311.700 3989.207 4641.298 3821.446 4818.576
Jan 1993       4732.398 4360.531 5112.881 4167.267 5317.689
Feb 1993       4780.196 4365.140 5205.907 4149.876 5435.446
Mar 1993       5219.170 4748.721 5702.169 4504.933 5962.773
Apr 1993       5164.246 4650.160 5693.522 4384.401 5979.640
May 1993       4949.949 4397.247 5521.041 4112.426 5830.514
Jun 1993       5035.976 4434.720 5658.674 4125.515 5996.633
Jul 1993       4584.535 3955.551 5239.447 3633.645 5596.136
Aug 1993       4839.699 4154.975 5553.574 3804.956 5942.701
Sep 1993       4868.346 4137.019 5632.826 3764.092 6050.245
Oct 1993       5044.941 4259.924 5866.888 3860.229 6316.151
Nov 1993       4632.659 3823.646 5484.664 3413.999 5952.052
Dec 1993       4504.345 3658.794 5398.417 3232.316 5890.095
Jan 1994       4918.287 4005.015 5883.362 3544.099 6413.884
Feb 1994       4955.395 3995.837 5971.870 3512.727 6531.481
\end{verbatim}

ARRUMAR O ERRO NO ETS\_TRANSF

COLOCAR INTERPRETAÇÃO

\hypertarget{resultados}{%
\section{Resultados}\label{resultados}}

apresente em tabelas e gráficos as previsões dos 4 modelos selecionados
e também apresente em uma tabela os resultados de acurácia dos 4 modelos
selecionados e dos modelos benchmarks. Comente os resultados de modo
objetivo;

\hypertarget{apuxeandice}{%
\section{Apêndice}\label{apuxeandice}}

Todo o projeto de composição deste documento pode ser encontrado aqui:
\url{https://github.com/cesar-galvao/trabalhos_series_temporais}



\end{document}
