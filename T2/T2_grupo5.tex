% Options for packages loaded elsewhere
\PassOptionsToPackage{unicode}{hyperref}
\PassOptionsToPackage{hyphens}{url}
\PassOptionsToPackage{dvipsnames,svgnames,x11names}{xcolor}
%
\documentclass[
  letterpaper,
  DIV=11,
  numbers=noendperiod]{scrartcl}

\usepackage{amsmath,amssymb}
\usepackage{iftex}
\ifPDFTeX
  \usepackage[T1]{fontenc}
  \usepackage[utf8]{inputenc}
  \usepackage{textcomp} % provide euro and other symbols
\else % if luatex or xetex
  \usepackage{unicode-math}
  \defaultfontfeatures{Scale=MatchLowercase}
  \defaultfontfeatures[\rmfamily]{Ligatures=TeX,Scale=1}
\fi
\usepackage{lmodern}
\ifPDFTeX\else  
    % xetex/luatex font selection
\fi
% Use upquote if available, for straight quotes in verbatim environments
\IfFileExists{upquote.sty}{\usepackage{upquote}}{}
\IfFileExists{microtype.sty}{% use microtype if available
  \usepackage[]{microtype}
  \UseMicrotypeSet[protrusion]{basicmath} % disable protrusion for tt fonts
}{}
\makeatletter
\@ifundefined{KOMAClassName}{% if non-KOMA class
  \IfFileExists{parskip.sty}{%
    \usepackage{parskip}
  }{% else
    \setlength{\parindent}{0pt}
    \setlength{\parskip}{6pt plus 2pt minus 1pt}}
}{% if KOMA class
  \KOMAoptions{parskip=half}}
\makeatother
\usepackage{xcolor}
\setlength{\emergencystretch}{3em} % prevent overfull lines
\setcounter{secnumdepth}{-\maxdimen} % remove section numbering
% Make \paragraph and \subparagraph free-standing
\ifx\paragraph\undefined\else
  \let\oldparagraph\paragraph
  \renewcommand{\paragraph}[1]{\oldparagraph{#1}\mbox{}}
\fi
\ifx\subparagraph\undefined\else
  \let\oldsubparagraph\subparagraph
  \renewcommand{\subparagraph}[1]{\oldsubparagraph{#1}\mbox{}}
\fi


\providecommand{\tightlist}{%
  \setlength{\itemsep}{0pt}\setlength{\parskip}{0pt}}\usepackage{longtable,booktabs,array}
\usepackage{calc} % for calculating minipage widths
% Correct order of tables after \paragraph or \subparagraph
\usepackage{etoolbox}
\makeatletter
\patchcmd\longtable{\par}{\if@noskipsec\mbox{}\fi\par}{}{}
\makeatother
% Allow footnotes in longtable head/foot
\IfFileExists{footnotehyper.sty}{\usepackage{footnotehyper}}{\usepackage{footnote}}
\makesavenoteenv{longtable}
\usepackage{graphicx}
\makeatletter
\def\maxwidth{\ifdim\Gin@nat@width>\linewidth\linewidth\else\Gin@nat@width\fi}
\def\maxheight{\ifdim\Gin@nat@height>\textheight\textheight\else\Gin@nat@height\fi}
\makeatother
% Scale images if necessary, so that they will not overflow the page
% margins by default, and it is still possible to overwrite the defaults
% using explicit options in \includegraphics[width, height, ...]{}
\setkeys{Gin}{width=\maxwidth,height=\maxheight,keepaspectratio}
% Set default figure placement to htbp
\makeatletter
\def\fps@figure{htbp}
\makeatother

\usepackage{booktabs}
\usepackage{longtable}
\usepackage{array}
\usepackage{multirow}
\usepackage{wrapfig}
\usepackage{float}
\usepackage{colortbl}
\usepackage{pdflscape}
\usepackage{tabu}
\usepackage{threeparttable}
\usepackage{threeparttablex}
\usepackage[normalem]{ulem}
\usepackage{makecell}
\usepackage{xcolor}
\usepackage[auth-lg]{authblk}
\KOMAoption{captions}{tableheading}
\makeatletter
\makeatother
\makeatletter
\makeatother
\makeatletter
\@ifpackageloaded{caption}{}{\usepackage{caption}}
\AtBeginDocument{%
\ifdefined\contentsname
  \renewcommand*\contentsname{Table of contents}
\else
  \newcommand\contentsname{Table of contents}
\fi
\ifdefined\listfigurename
  \renewcommand*\listfigurename{List of Figures}
\else
  \newcommand\listfigurename{List of Figures}
\fi
\ifdefined\listtablename
  \renewcommand*\listtablename{List of Tables}
\else
  \newcommand\listtablename{List of Tables}
\fi
\ifdefined\figurename
  \renewcommand*\figurename{Figure}
\else
  \newcommand\figurename{Figure}
\fi
\ifdefined\tablename
  \renewcommand*\tablename{Table}
\else
  \newcommand\tablename{Table}
\fi
}
\@ifpackageloaded{float}{}{\usepackage{float}}
\floatstyle{ruled}
\@ifundefined{c@chapter}{\newfloat{codelisting}{h}{lop}}{\newfloat{codelisting}{h}{lop}[chapter]}
\floatname{codelisting}{Listing}
\newcommand*\listoflistings{\listof{codelisting}{List of Listings}}
\makeatother
\makeatletter
\@ifpackageloaded{caption}{}{\usepackage{caption}}
\@ifpackageloaded{subcaption}{}{\usepackage{subcaption}}
\makeatother
\makeatletter
\@ifpackageloaded{tcolorbox}{}{\usepackage[skins,breakable]{tcolorbox}}
\makeatother
\makeatletter
\@ifundefined{shadecolor}{\definecolor{shadecolor}{rgb}{.97, .97, .97}}
\makeatother
\makeatletter
\makeatother
\makeatletter
\makeatother
\ifLuaTeX
  \usepackage{selnolig}  % disable illegal ligatures
\fi
\IfFileExists{bookmark.sty}{\usepackage{bookmark}}{\usepackage{hyperref}}
\IfFileExists{xurl.sty}{\usepackage{xurl}}{} % add URL line breaks if available
\urlstyle{same} % disable monospaced font for URLs
\hypersetup{
  pdftitle={Trabalho Prático 2},
  pdfauthor={Ana Carolina Vianna - 18/0097261; César Augusto Galvão - 19/0011572; Yan Flávio Vianna - 14/0166149},
  colorlinks=true,
  linkcolor={blue},
  filecolor={Maroon},
  citecolor={Blue},
  urlcolor={Blue},
  pdfcreator={LaTeX via pandoc}}

\title{Trabalho Prático 2}
\usepackage{etoolbox}
\makeatletter
\providecommand{\subtitle}[1]{% add subtitle to \maketitle
  \apptocmd{\@title}{\par {\large #1 \par}}{}{}
}
\makeatother
\subtitle{Análise de Séries Temporais - 1/2023}
\author{Ana Carolina Vianna - 18/0097261 \and César Augusto Galvão -
19/0011572 \and Yan Flávio Vianna - 14/0166149}
\date{}

\begin{document}
\maketitle
\ifdefined\Shaded\renewenvironment{Shaded}{\begin{tcolorbox}[boxrule=0pt, interior hidden, breakable, borderline west={3pt}{0pt}{shadecolor}, sharp corners, enhanced, frame hidden]}{\end{tcolorbox}}\fi

\renewcommand*\contentsname{Table of contents}
{
\hypersetup{linkcolor=}
\setcounter{tocdepth}{2}
\tableofcontents
}
\newpage{}

\hypertarget{introduuxe7uxe3o-suxe9rie-selecionada-caracteruxedsticas-e-decomposiuxe7uxe3o}{%
\section{Introdução: série selecionada, características e
decomposição}\label{introduuxe7uxe3o-suxe9rie-selecionada-caracteruxedsticas-e-decomposiuxe7uxe3o}}

A série temporal escolhida foi a de número \emph{id} correspondente a
2183. De acordo com a definição do próprio pacote, refere-se a
\emph{Fluid power shipments - hydraulic index}. Foram realizadas medidas
mensais de 1983 a 1992 e o horizonte de previsão requerido é das 18
ocorrências seguintes.

O gráfico da série, com \emph{in} e \emph{out-sample}, é exposto a
seguir.

\includegraphics{T2_grupo5_files/figure-pdf/plot-serie-total-1.pdf}

A série aparenta ter dois períodos, pelo menos: um ciclo anual e outro
que compreende um período maior. No entanto, ao se tentar decompor a
série com múltiplas sazonalidades, obté-se o seguinte:

\begin{itemize}
\tightlist
\item
  \textbf{Adicionando uma componente sazonal com ciclo menor que 1 ano}
  -- uma das componentes sazonais apresenta heteroscedasticidade;
\item
  \textbf{Adicionando uma componente sazonal com ciclo maior que 1 ano}
  -- resíduos apresentam periodicidade ou heteroscedasticidade.
\end{itemize}

Optou-se portanto pela decomposição STL (apesar de os dados terem
inicialmente formado um objeto \texttt{msts}) apenas com a sazonalidade
anual, mas fica evidente que esta decomposição não é adequada quando se
avalia a componente de tendência, que aparenta ainda carregar algum
componente periódico. Os resíduos aparentam um comportamento aleatório e
têm média -0.104, o que é próximo de zero o suficiente considerando a
magnitude dos dados da série. A decomposição é exposta a seguir.

\includegraphics{T2_grupo5_files/figure-pdf/grafico-mstl-1.pdf}

\hypertarget{modelos-arima-seleuxe7uxe3o-transformauxe7uxf5es-e-resuxedduos}{%
\section{Modelos ARIMA: seleção, transformações e
resíduos}\label{modelos-arima-seleuxe7uxe3o-transformauxe7uxf5es-e-resuxedduos}}

\hypertarget{modelo-sem-transformauxe7uxe3o}{%
\subsection{Modelo sem
transformação}\label{modelo-sem-transformauxe7uxe3o}}

\hypertarget{seleuxe7uxe3o}{%
\subsubsection{Seleção}\label{seleuxe7uxe3o}}

Primeiramente, utilizou-se as funções \emph{ndiffs()} e \emph{nsdiffs()}
do pacote \emph{forecast} para identificar quantas diferenças simples e
sazonais seriam necessárias para que a série se tornasse estacionária.
Concluiu-se pelo resultado dessas funções que são necessárias uma
diferenciação simples e uma sazonal. O teste KPSS confirma isso ao não
rejeitar a hipótese nula de estacionariedade da série (com diferenças já
aplicadas) ao nível de 5\% de significância.

\begin{longtable*}{lcc}
\toprule
 & Estatística & p-valor\\
\midrule
\endfirsthead
\multicolumn{3}{@{}l}{\textit{(continued)}}\\
\toprule
 & Estatística & p-valor\\
\midrule
\endhead

\endfoot
\bottomrule
\endlastfoot
\cellcolor{gray!15}{KPSS Test for Level Stationarity} & \cellcolor{gray!15}{0.11} & \cellcolor{gray!15}{0.1}\\*
\end{longtable*}

Prosseguimos com a seleção do melhor modelo ARIMA avaliando os gráficos
de ACF e PACF. O primeiro parece apresentar quebra no primeiro lag
sazonal, enquanto o segundo tem quebra no segundo lag simples,
configurando um \(\text{ARIMA}(2,1,0)\times(0,1,1)_{12}\) (porém os
resíduos para este modelo não ficam muito bons). Entretanto, como não
fica nítido um comportamento de queda amortizada, preferiu-se utilizar
outro critério para a seleção do modelo.

\includegraphics{T2_grupo5_files/figure-pdf/acf-pacf-sem-transformacao-1.pdf}

Optou-se pela varredura de combinações de \(p\), \(q\), \(P\) e \(Q\),
com \(d\) e \(D\) fixados em 1, como resultado das diferenciações ja
avaliadas. Utilizando o critério de Akaike corrigido, seleciona-se o
modelo \(\text{ARIMA}(2,1,2)\times(0,1,2)_{12}\) para a série, que
possui o menor escore entre os modelos testados.

Ao se utilizar a função \texttt{auto.arima()}, recebe-se um modelo
sugerido \(\text{ARIMA}(2,1,2)\times(2,1,0)_{12}\), porém com AICc
superior àquele identificado na varredura. Opta-se pelo modelo
selecionado manualmente.

\hypertarget{resuxedduos}{%
\subsubsection{Resíduos}\label{resuxedduos}}

Foram retirados os zeros da inicialização para possibilitar a análise
dos resíduos. Observa-se pelo gráfico que os resíduos são aleatórios e
aparentemente centrados em zero, com variação constante. Além disso,
verifica-se uma distribuição aproximadamente normal, mas com caudas mais
pesadas. Finalmente, o gráfico ACF apresenta que a autocorrelação dos
resíduos está, em sua grande maioria, dentro da banda de confiança, com
exceção de um ponto, que extrapola ligeiramente a margem.

\includegraphics{T2_grupo5_files/figure-pdf/residuos-arima-1.pdf}

Por fim, realiza-se testes de hipótese para independência e normalidade
(o teste KPSS para estacionariedade já foi apresentado) e seus
resultados são apresentados na tabela a seguir. De fato, o teste de
Shapiro-Wilk não rejeita a normalidade da distribuição dos resíduos
apesar de o gráfico QQ apresentar caudas pesadas. Além disso, o teste
Ljung-Box com \emph{lag} igual a 15 também não rejeita a independência
entre os resíduos e, consequentemente, os dados da série.

\begin{longtable*}{lccc}
\toprule
 & Estatística & p-valor & Lag\\
\midrule
\endfirsthead
\multicolumn{4}{@{}l}{\textit{(continued)}}\\
\toprule
 & Estatística & p-valor & Lag\\
\midrule
\endhead

\endfoot
\bottomrule
\endlastfoot
\cellcolor{gray!15}{Box-Ljung test} & \cellcolor{gray!15}{8.90} & \cellcolor{gray!15}{0.88} & \cellcolor{gray!15}{15}\\
Shapiro-Wilk normality test & 0.99 & 0.35 & \\*
\end{longtable*}

\hypertarget{modelo-com-transformauxe7uxe3o}{%
\subsection{Modelo com
transformação}\label{modelo-com-transformauxe7uxe3o}}

\hypertarget{seleuxe7uxe3o-1}{%
\subsubsection{Seleção}\label{seleuxe7uxe3o-1}}

Foi utilizada a função \emph{BoxCox.lambda()} do pacote \emph{forecast}
para decidir de forma automatizada o melhor valor de lambda para a
transformação de Box-Cox. A função sugere um valor de \(\lambda =\)
0.71.

Apesar de haver uma sugestão de transformação, não é possível avaliar
graficamente se houve uma diferença significativa no comportamento da
série temporal excetuando-se a escala, como se pode ver nos eixos dos
gráficos a seguir.

\includegraphics{T2_grupo5_files/figure-pdf/comparacao-transformacao-arima-1.pdf}

Após aplicar a tranformação de Box-Cox na série, utilizou-se as funções
\emph{ndiffs()} e \emph{nsdiffs()} para identificar quantas
diferenciações simples e sazonais seriam necessárias para que a série se
torne estacionária. Concluiu-se que são necessárias uma diferenciações
simples e uma diferenciações sazonal, o que é confirmado pelo resultado
do teste KPSS nos resíduos da série com as diferenças já aplicadas.

\begin{longtable*}{lcc}
\toprule
 & Estatística & p-valor\\
\midrule
\endfirsthead
\multicolumn{3}{@{}l}{\textit{(continued)}}\\
\toprule
 & Estatística & p-valor\\
\midrule
\endhead

\endfoot
\bottomrule
\endlastfoot
\cellcolor{gray!15}{KPSS Test for Level Stationarity} & \cellcolor{gray!15}{0.12} & \cellcolor{gray!15}{0.1}\\*
\end{longtable*}

O gráfico da ACF parece apresentar quebra no primeiro lag sazonal,
enquanto o PACF tem quebra no segundo lag simples, o que configura um
\(\text{ARIMA}(2,1,0)\times(0,1,1)_{12}\) (porém, mais uma vez, os
resíduos para este modelo não ficam muito bons). Entretanto, os gráficos
não evidenciam comportamentos claros para a série. Novamente, os
resíduos parecem ter média igual a zero.

\includegraphics{T2_grupo5_files/figure-pdf/acf-pacf-arima-boxcox-1.pdf}

Foram testadas combinações de \(p\), \(q\), \(P\) e \(Q\), com \(d\) e
\(D\) fixados em 1 e, em seguida, selecionou-se o modelo ARIMA que
apresentava menor valor do AICc. Temos, então, que o modelo escolhido
para a série transformada é um
\(\text{ARIMA}(2,1,2)\times(0,1,2)_{12}\), assim como no caso da série
sem transformação. Utilizando-se a função \texttt{auto.arima()}
recebe-se uma sugestão de um modelo \(ARIMA(3,1,1)\times(2,1,0)_{12}\)
mas, assim como ocorre no modelo sem transformação, opta-se pelo modelo
selecionado manualmente por apresentar um AICc menor.

\hypertarget{resuxedduos-1}{%
\subsubsection{Resíduos}\label{resuxedduos-1}}

Foram retirados os zeros da inicialização para seguir com a análise dos
resíduos. O gráfico da série dos resíduos sugere aleatoriedade e o QQ
plot distribuição aproximadamente normal. Por último, o gráfico ACF
mostra que a autocorrelação dos resíduos está dentro da banda de
confiança, com exceção de um ponto que excede um pouco este limite.

\includegraphics{T2_grupo5_files/figure-pdf/residuos-arima-boxcox-1.pdf}

Assim como ocorre para a série não transformada, os testes de
Shapiro-Wilk e Ljung-box com \emph{lag} igual a 15 não apresentam
indicação para rejeição de suas hipóteses nulas. Isto é, pode-se dizer
que a série transformada tem distribuição normal e seus resíduos são
independentes.

\begin{longtable*}{lccc}
\toprule
 & Estatística & p-valor & Lag\\
\midrule
\endfirsthead
\multicolumn{4}{@{}l}{\textit{(continued)}}\\
\toprule
 & Estatística & p-valor & Lag\\
\midrule
\endhead

\endfoot
\bottomrule
\endlastfoot
\cellcolor{gray!15}{Box-Ljung test} & \cellcolor{gray!15}{8.41} & \cellcolor{gray!15}{0.91} & \cellcolor{gray!15}{15}\\
Shapiro-Wilk normality test & 0.98 & 0.23 & \\*
\end{longtable*}

\hypertarget{modelos-ets-seleuxe7uxe3o-transformauxe7uxf5es-e-resuxedduos}{%
\section{Modelos ETS: seleção, transformações e
resíduos}\label{modelos-ets-seleuxe7uxe3o-transformauxe7uxf5es-e-resuxedduos}}

\hypertarget{modelo-sem-transformauxe7uxe3o-1}{%
\subsection{Modelo sem
transformação}\label{modelo-sem-transformauxe7uxe3o-1}}

\hypertarget{seleuxe7uxe3o-2}{%
\subsubsection{Seleção}\label{seleuxe7uxe3o-2}}

Para a seleção do modelo ETS, foi realizada uma varredura com todas as
combinações possíveis de erro, tendência e sazonalidade, assim como a
aplicação ou não de \emph{damp} na tendência. Os seis modelos com
melhores indicadores são exibidos na tabela a seguir para comparação.

\begin{longtable*}{lccc}
\toprule
Modelo & AIC & AICc & BIC\\
\midrule
\endfirsthead
\multicolumn{4}{@{}l}{\textit{(continued)}}\\
\toprule
Modelo & AIC & AICc & BIC\\
\midrule
\endhead

\endfoot
\bottomrule
\endlastfoot
\cellcolor{gray!15}{ETS(M,Ad,M)} & \cellcolor{gray!15}{1761.30} & \cellcolor{gray!15}{1768.36} & \cellcolor{gray!15}{1810.87}\\
ETS(M,M,M) & 1761.94 & 1769.00 & 1811.51\\
\cellcolor{gray!15}{ETS(A,Ad,A)} & \cellcolor{gray!15}{1764.25} & \cellcolor{gray!15}{1771.30} & \cellcolor{gray!15}{1813.81}\\
ETS(M,Ad,A) & 1767.73 & 1774.78 & 1817.29\\
\cellcolor{gray!15}{ETS(M,A,M)} & \cellcolor{gray!15}{1769.04} & \cellcolor{gray!15}{1775.29} & \cellcolor{gray!15}{1815.86}\\
ETS(A,A,A) & 1771.20 & 1777.44 & 1818.01\\*
\end{longtable*}

De fato, modelos que misturam termos aditivos e multiplicativos
apresentam os melhores indicadores AICc, mas o desempenho da função
\texttt{ets()} é instável nessas condições. Por isso, opta-se pelo uso
do terceiro melhor modelo, \(ETS(A, Ad, A)\). A decomposição da série
temporal analisada utilizando este modelo é exposta no gráfico a seguir.

\includegraphics{T2_grupo5_files/figure-pdf/melhor-fit-ETL-sem-transf-1.pdf}

\hypertarget{resuxedduos-2}{%
\subsubsection{Resíduos}\label{resuxedduos-2}}

Uma análise visual dos resíduos indica comportamento aleatório em torno
de zero e autocorrelações próximas a zero. Quanto à distribuição, a
amostra parece ter uma distribuição próxima à normal, mas com caudas
mais pesadas.

\begin{figure}

{\centering \includegraphics{T2_grupo5_files/figure-pdf/residuos-ets-sem-transform-1.pdf}

}

\end{figure}

Observa-se ainda pelo resultado dos testes de hipótese a seguir temos
uma série estacionária, com resíduos normalmente distribuídos e
mutuamente independentes.

\begin{longtable*}{lccc}
\toprule
 & Estatística & p-valor & Lag\\
\midrule
\endfirsthead
\multicolumn{4}{@{}l}{\textit{(continued)}}\\
\toprule
 & Estatística & p-valor & Lag\\
\midrule
\endhead

\endfoot
\bottomrule
\endlastfoot
\cellcolor{gray!15}{KPSS Test for Level Stationarity} & \cellcolor{gray!15}{0.07} & \cellcolor{gray!15}{0.10} & \cellcolor{gray!15}{4}\\
Box-Ljung test & 21.35 & 0.13 & 15\\
\cellcolor{gray!15}{Shapiro-Wilk normality test} & \cellcolor{gray!15}{0.99} & \cellcolor{gray!15}{0.38} & \cellcolor{gray!15}{}\\*
\end{longtable*}

\hypertarget{modelo-com-transformauxe7uxe3o-1}{%
\subsection{Modelo com
transformação}\label{modelo-com-transformauxe7uxe3o-1}}

\hypertarget{seleuxe7uxe3o-3}{%
\subsubsection{Seleção}\label{seleuxe7uxe3o-3}}

Utilizando a função \texttt{BoxCox.lambda()} obtém-se uma sugestão de
transformação com \(\lambda =\) 0.712. A série transformada é exibida no
gráfico a seguir.

\includegraphics{T2_grupo5_files/figure-pdf/ETS-com-transf-1.pdf}

No caso do modelo com transformação BoxCox a recomendação de modelo,
obtida pelo mesmo método de varredura, é primariamente a seleção já
feita para o modelo sem transformação, que é o \(ETS(A, Ad, A)\).

\begin{longtable*}{lccc}
\toprule
Modelo transformado & AIC & AICc & BIC\\
\midrule
\endfirsthead
\multicolumn{4}{@{}l}{\textit{(continued)}}\\
\toprule
Modelo transformado & AIC & AICc & BIC\\
\midrule
\endhead

\endfoot
\bottomrule
\endlastfoot
\cellcolor{gray!15}{ETS(A,Ad,A)} & \cellcolor{gray!15}{1205.77} & \cellcolor{gray!15}{1212.82} & \cellcolor{gray!15}{1255.33}\\
ETS(M,M,M) & 1207.69 & 1214.75 & 1257.26\\
\cellcolor{gray!15}{ETS(M,Ad,M)} & \cellcolor{gray!15}{1207.92} & \cellcolor{gray!15}{1214.98} & \cellcolor{gray!15}{1257.49}\\
ETS(M,Ad,A) & 1208.24 & 1215.29 & 1257.81\\
\cellcolor{gray!15}{ETS(A,A,A)} & \cellcolor{gray!15}{1218.37} & \cellcolor{gray!15}{1224.62} & \cellcolor{gray!15}{1265.18}\\
ETS(M,A,A) & 1221.45 & 1227.69 & 1268.26\\*
\end{longtable*}

Os componentes do ajuste do modelo \(ETS(A, Ad, A)\) são expostas a
seguir e, como no caso anterior, a transformação não apresenta nenhuma
mudança aparente no comportamento da série.

\includegraphics{T2_grupo5_files/figure-pdf/decomposicao-ets-com-transformacao-1.pdf}

\hypertarget{resuxedduos-3}{%
\subsubsection{Resíduos}\label{resuxedduos-3}}

Assim como ocorre para a série não transformada, os resíduos parecem
exibir comportamento aleatório em torno de zero, autocorrelações
próximas a zero e distribuição próxima à normal, mas com caudas mais
pesadas.

\begin{figure}

{\centering \includegraphics{T2_grupo5_files/figure-pdf/resíduos-modelo-transformado-1.pdf}

}

\end{figure}

Novamente, os testes de hipótese expostos na tabela a seguir corroboram
a hipótese de se tratar de uma série estacionária, com resíduos
normalmente distribuídos e mutuamente independentes.

\begin{longtable*}{lccc}
\toprule
 & Estatística & p-valor & Lag\\
\midrule
\endfirsthead
\multicolumn{4}{@{}l}{\textit{(continued)}}\\
\toprule
 & Estatística & p-valor & Lag\\
\midrule
\endhead

\endfoot
\bottomrule
\endlastfoot
\cellcolor{gray!15}{KPSS Test for Level Stationarity} & \cellcolor{gray!15}{0.07} & \cellcolor{gray!15}{0.10} & \cellcolor{gray!15}{4}\\
Box-Ljung test & 20.48 & 0.15 & 15\\
\cellcolor{gray!15}{Shapiro-Wilk normality test} & \cellcolor{gray!15}{0.98} & \cellcolor{gray!15}{0.16} & \cellcolor{gray!15}{}\\*
\end{longtable*}

\hypertarget{estudo-de-desempenho-preditivo}{%
\section{Estudo de desempenho
preditivo}\label{estudo-de-desempenho-preditivo}}

Para realizar a análise do desempenho preditivo usando uma abordagem de
janela deslizante, o estudo considera uma janela de tamanho \(n-14\) e
calcula os erros de previsão para horizontes de até 5 períodos.
Utilizando os modelos previamente mencionados para criar as funções de
previsão, os resultados são apresentados em um gráfico e uma tabela,
mostrando os erros absolutos para cada horizonte de previsão.

\hypertarget{resultados-da-janela-deslizante}{%
\subsection{Resultados da Janela
Deslizante}\label{resultados-da-janela-deslizante}}

\begin{longtable*}{ccccc}
\toprule
 & ARIMA & ETS & ARIMA Transformada & ETS Transformada\\
\midrule
\endfirsthead
\multicolumn{5}{@{}l}{\textit{(continued)}}\\
\toprule
 & ARIMA & ETS & ARIMA Transformada & ETS Transformada\\
\midrule
\endhead

\endfoot
\bottomrule
\endlastfoot
\cellcolor{gray!15}{h=1} & \cellcolor{gray!15}{130.701} & \cellcolor{gray!15}{126.428} & \cellcolor{gray!15}{122.244} & \cellcolor{gray!15}{124.525}\\
h=2 & 133.301 & 128.668 & 144.609 & 136.707\\
\cellcolor{gray!15}{h=3} & \cellcolor{gray!15}{128.709} & \cellcolor{gray!15}{126.408} & \cellcolor{gray!15}{163.981} & \cellcolor{gray!15}{167.741}\\
h=4 & 136.742 & 128.690 & 180.380 & 181.928\\
\cellcolor{gray!15}{h=5} & \cellcolor{gray!15}{173.517} & \cellcolor{gray!15}{161.675} & \cellcolor{gray!15}{210.707} & \cellcolor{gray!15}{210.361}\\*
\end{longtable*}

\hypertarget{performance-em-relauxe7uxe3o-aos-horizontes-de-previsuxe3o}{%
\subsection{Performance em relação aos horizontes de
previsão}\label{performance-em-relauxe7uxe3o-aos-horizontes-de-previsuxe3o}}

\begin{figure}

{\centering \includegraphics{T2_grupo5_files/figure-pdf/performance-1.pdf}

}

\end{figure}

Ao analisar o gráfico obtido, observa-se que o modelo ARIMA, tanto para
a série original quanto para a série transformada, apresentou erros
médios menores na maioria dos horizontes de previsão, com exceção do
horizonte 1. Portanto, o modelo
\(\text{ARIMA}(2,1,2)\times(2,1,0)_{12}\) obteve melhor comportamento
para o caso original e o caso transformado.

\hypertarget{gruxe1ficos-da-previsuxe3o-pontual-e-da-previsuxe3o-intervalar-dos-4-modelos-selecionados}{%
\section{Gráficos da previsão pontual e da previsão intervalar dos 4
modelos
selecionados}\label{gruxe1ficos-da-previsuxe3o-pontual-e-da-previsuxe3o-intervalar-dos-4-modelos-selecionados}}

Utilizou-se o modelo \(\text{ARIMA}(2,1,2)\times(2,1,0)_{12}\) para
realizar previsões pontuais e intervalares dos modelos selecionados. O
horizonte de previsão fornecido pelo banco de dados foi de 18 pontos.

Os gráficos fornecidos permitem visualizar as previsões pontuais e
intervalares da série temporal da competição de previsão M3, com uma
probabilidade de cobertura de 95\%. Já as tabelas, apresentam as
previsões para um horizonte de 18 pontos e fornecem os intervalos das
probabilidades de cobertura 80\% e 95\%.

Ao analisar os gráficos e as tabelas nas seções a seguir, observa-se que
em todos eles a previsão está aparentemente ajustada à série original e
os intervalos de probabilidades de cobertura analisados possuem
aproximadamente o mesmo espectro.

\newpage{}

\hypertarget{arima} & \multicolumn{2}{c}{IC para 95\%} \\
\cmidrule(l{3pt}r{3pt}){3-4} \cmidrule(l{3pt}r{3pt}){5-6}
Período (mês/ano) & Prev. Pontual & LI & LS & LI & LS\\
\midrule
Sep 1992 & 4795.36 & 4568.19 & 5022.53 & 4447.93 & 5142.79\\
Oct 1992 & 5016.38 & 4763.01 & 5269.75 & 4628.88 & 5403.88\\
Nov 1992 & 4541.28 & 4248.64 & 4833.91 & 4093.73 & 4988.82\\
Dec 1992 & 4391.44 & 4032.32 & 4750.56 & 3842.21 & 4940.67\\
Jan 1993 & 4848.16 & 4440.26 & 5256.06 & 4224.33 & 5471.99\\
Feb 1993 & 4905.30 & 4467.19 & 5343.41 & 4235.26 & 5575.33\\
Mar 1993 & 5329.73 & 4861.35 & 5798.10 & 4613.41 & 6046.05\\
Apr 1993 & 5203.39 & 4700.48 & 5706.30 & 4434.25 & 5972.52\\
May 1993 & 5052.67 & 4518.13 & 5587.20 & 4235.17 & 5870.16\\
Jun 1993 & 5245.42 & 4683.93 & 5806.91 & 4386.70 & 6104.14\\
Jul 1993 & 4859.08 & 4271.73 & 5446.43 & 3960.81 & 5757.35\\
Aug 1993 & 5014.81 & 4401.33 & 5628.30 & 4076.57 & 5953.06\\
Sep 1993 & 5125.55 & 4459.12 & 5791.98 & 4106.34 & 6144.77\\
Oct 1993 & 5328.57 & 4627.99 & 6029.14 & 4257.13 & 6400.00\\
Nov 1993 & 4908.79 & 4172.04 & 5645.55 & 3782.02 & 6035.57\\
Dec 1993 & 4788.71 & 4010.22 & 5567.20 & 3598.12 & 5979.30\\
Jan 1994 & 5210.32 & 4393.63 & 6027.01 & 3961.30 & 6459.34\\
Feb 1994 & 5256.72 & 4407.77 & 6105.66 & 3958.37 & 6555.07\\
\bottomrule
\end{longtable*}

\newpage{}

\hypertarget{arima-com-tranformauxe7uxe3o} & \multicolumn{2}{c}{IC para 95\%} \\
\cmidrule(l{3pt}r{3pt}){3-4} \cmidrule(l{3pt}r{3pt}){5-6}
Período (mês/ano) & Prev. Pontual & LI & LS & LI & LS\\
\midrule
Sep 1992 & 4799.75 & 4565.17 & 5037.69 & 4442.37 & 5164.97\\
Oct 1992 & 5035.24 & 4767.61 & 5307.02 & 4627.65 & 5452.55\\
Nov 1992 & 4537.15 & 4237.28 & 4842.83 & 4080.96 & 5006.95\\
Dec 1992 & 4379.43 & 4016.24 & 4751.52 & 3827.68 & 4951.99\\
Jan 1993 & 4854.19 & 4428.23 & 5291.21 & 4207.36 & 5526.88\\
Feb 1993 & 4922.09 & 4462.59 & 5394.30 & 4224.66 & 5649.25\\
Mar 1993 & 5364.43 & 4861.28 & 5881.57 & 4600.79 & 6160.80\\
Apr 1993 & 5235.77 & 4700.00 & 5787.82 & 4423.23 & 6086.42\\
May 1993 & 5077.22 & 4512.97 & 5660.16 & 4222.16 & 5976.02\\
Jun 1993 & 5275.45 & 4676.15 & 5895.06 & 4367.47 & 6230.95\\
Jul 1993 & 4867.74 & 4256.48 & 5501.96 & 3942.66 & 5846.60\\
Aug 1993 & 5035.62 & 4391.14 & 5704.81 & 4060.48 & 6068.61\\
Sep 1993 & 5155.40 & 4452.82 & 5886.74 & 4093.16 & 6284.98\\
Oct 1993 & 5372.07 & 4624.76 & 6150.63 & 4242.50 & 6574.82\\
Nov 1993 & 4923.69 & 4159.71 & 5723.50 & 3770.67 & 6160.63\\
Dec 1993 & 4797.72 & 3998.70 & 5637.11 & 3593.16 & 6096.87\\
Jan 1994 & 5246.60 & 4386.32 & 6149.64 & 3949.35 & 6644.03\\
Feb 1994 & 5302.61 & 4406.47 & 6244.74 & 3951.96 & 6761.01\\
\bottomrule
\end{longtable*}

\newpage{}

\hypertarget{ets} & \multicolumn{2}{c}{IC para 95\%} \\
\cmidrule(l{3pt}r{3pt}){3-4} \cmidrule(l{3pt}r{3pt}){5-6}
Período (mês/ano) & Prev. Pontual & LI & LS & LI & LS\\
\midrule
Sep 1992 & 4613.02 & 4391.65 & 4834.40 & 4274.46 & 4951.59\\
Oct 1992 & 4808.08 & 4557.39 & 5058.77 & 4424.68 & 5191.48\\
Nov 1992 & 4437.83 & 4153.58 & 4722.07 & 4003.11 & 4872.54\\
Dec 1992 & 4337.83 & 4016.80 & 4658.86 & 3846.85 & 4828.80\\
Jan 1993 & 4746.47 & 4386.22 & 5106.72 & 4195.51 & 5297.43\\
Feb 1993 & 4771.43 & 4370.15 & 5172.71 & 4157.73 & 5385.14\\
Mar 1993 & 5198.16 & 4754.52 & 5641.80 & 4519.66 & 5876.65\\
Apr 1993 & 5139.63 & 4652.66 & 5626.60 & 4394.88 & 5884.39\\
May 1993 & 4950.91 & 4419.94 & 5481.88 & 4138.87 & 5762.96\\
Jun 1993 & 5019.38 & 4443.96 & 5594.80 & 4139.35 & 5899.41\\
Jul 1993 & 4587.10 & 3966.96 & 5207.23 & 3638.68 & 5535.51\\
Aug 1993 & 4840.20 & 4175.23 & 5505.18 & 3823.21 & 5857.20\\
Sep 1993 & 4844.61 & 4134.77 & 5554.46 & 3759.00 & 5930.23\\
Oct 1993 & 5024.39 & 4269.79 & 5778.99 & 3870.32 & 6178.45\\
Nov 1993 & 4639.86 & 3840.68 & 5439.05 & 3417.61 & 5862.11\\
Dec 1993 & 4526.53 & 3682.99 & 5370.07 & 3236.45 & 5816.61\\
Jan 1994 & 4922.72 & 4035.12 & 5810.33 & 3565.25 & 6280.20\\
Feb 1994 & 4936.06 & 4004.71 & 5867.40 & 3511.69 & 6360.43\\
\bottomrule
\end{longtable*}

\newpage{}

\hypertarget{ets-com-transformauxe7uxe3o} & \multicolumn{2}{c}{IC para 95\%} \\
\cmidrule(l{3pt}r{3pt}){3-4} \cmidrule(l{3pt}r{3pt}){5-6}
Período (mês/ano) & Prev. Pontual & LI & LS & LI & LS\\
\midrule
Sep 1992 & 4631.45 & 4406.50 & 4859.59 & 4288.73 & 4981.63\\
Oct 1992 & 4819.80 & 4560.49 & 5083.19 & 4424.91 & 5224.24\\
Nov 1992 & 4426.13 & 4138.59 & 4719.16 & 3988.65 & 4876.45\\
Dec 1992 & 4311.70 & 3989.21 & 4641.30 & 3821.45 & 4818.58\\
Jan 1993 & 4732.40 & 4360.53 & 5112.88 & 4167.27 & 5317.69\\
Feb 1993 & 4780.20 & 4365.14 & 5205.91 & 4149.88 & 5435.45\\
Mar 1993 & 5219.17 & 4748.72 & 5702.17 & 4504.93 & 5962.77\\
Apr 1993 & 5164.25 & 4650.16 & 5693.52 & 4384.40 & 5979.64\\
May 1993 & 4949.95 & 4397.25 & 5521.04 & 4112.43 & 5830.51\\
Jun 1993 & 5035.98 & 4434.72 & 5658.67 & 4125.51 & 5996.63\\
Jul 1993 & 4584.53 & 3955.55 & 5239.45 & 3633.64 & 5596.14\\
Aug 1993 & 4839.70 & 4154.97 & 5553.57 & 3804.96 & 5942.70\\
Sep 1993 & 4868.35 & 4137.02 & 5632.83 & 3764.09 & 6050.25\\
Oct 1993 & 5044.94 & 4259.92 & 5866.89 & 3860.23 & 6316.15\\
Nov 1993 & 4632.66 & 3823.65 & 5484.66 & 3414.00 & 5952.05\\
Dec 1993 & 4504.35 & 3658.79 & 5398.42 & 3232.32 & 5890.09\\
Jan 1994 & 4918.29 & 4005.02 & 5883.36 & 3544.10 & 6413.88\\
Feb 1994 & 4955.39 & 3995.84 & 5971.87 & 3512.73 & 6531.48\\
\bottomrule
\end{longtable*}

\newpage{}

\hypertarget{resultados}{%
\section{Resultados}\label{resultados}}

apresente em tabelas e gráficos as previsões dos 4 modelos selecionados
e também apresente em uma tabela os resultados de acurácia dos 4 modelos
selecionados e dos modelos benchmarks. Comente os resultados de modo
objetivo;

\hypertarget{apuxeandice}{%
\section{Apêndice}\label{apuxeandice}}

Todo o projeto de composição deste documento pode ser encontrado aqui:
\url{https://github.com/cesar-galvao/trabalhos_series_temporais}



\end{document}
